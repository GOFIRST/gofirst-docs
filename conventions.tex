\documentclass{article}

\usepackage{fullpage}
\usepackage{pdfsync}
\usepackage{url}
\usepackage[T1]{fontenc}

\begin{document}
\title{GO FIRST Source Code Conventions}
\author{Max Veit}
\date{Last Updated: \today}

\maketitle

\section{References}
\begin{itemize}
    \item Sun Code Conventons for Java: \url{http://www.oracle.com/technetwork/java/codeconventions-150003.pdf} \\
        This is a good, but not perfect, introduction to basic style for Java source code. We will be using it with a few modifications (detailed below) for all of our Java, C and C++ code because of the syntactic similarities between the languages. Please read this thoroughly if you have no experience with C syntax (come on, it's only 18 pages. A TL;DR response will disqualify you from our programming team). If you have programmed in the C-syntax languages before, at least skim the guide to see what specific conventions we'll be using.
    \item Style Guide for Python Code: \url{http://www.python.org/dev/peps/pep-0008/} \\
        A very good guide for Python code with general advice that applies to other programming languages as well. Please at least read the introduction and section on comments. If we do any coding in Python, the style should follow this guide. Comments in other languages should follow the advice in this guide; for all other areas the language's style guide naturally has precedence.
    \item Pasta Code: \url{http://www.gnu.org/fun/jokes/pasta.code.html} \\
        If you're doing object-oriented programming, please try for ravioli code. Canenderli code and Ristto code should be avoided because of poor maintainability. Polenta code is nice, too, if it never needs to be modified (which is generally never the case, so watch out!)
    \item Good Code: \url{http://xkcd.com/844/} \\
        Please try to stay in the "Code Well" loop. Yes, requirements do change, and I've had to throw out a few projects because of this, but trust me: this is not always the case!
\end{itemize}

\end{document}

